\documentclass[a4paper,oneside]{article}

\title{SCEZ Library Documentation - Old version}
\author{Matthias Bruestle}

\begin{document}

\maketitle

\tableofcontents

\section{Include Files}

%----------------------------------------------------------------------

\subsection{sc\_general.h}

\begin{verbatim}
#if defined( WITH_CRYPTLIB )
#include "../crypt.h"
#else
#define BYTE	unsigned char
#define	CHAR	signed char
#define	WORD	unsigned short
#define	LONG	unsigned int
#define BOOLEAN	int
#define FALSE	0
#define TRUE	!0
#ifndef max
  #define max( a, b )   ( ( ( a ) > ( b ) ) ? ( ( int ) a ) : ( ( int ) b ) )
#endif /* !max */
#ifndef min
  #define min( a, b )   ( ( ( a ) < ( b ) ) ? ( ( int ) a ) : ( ( int ) b ) )
#endif /* !min */
#endif /* WITH_CRYPTLIB */

#include <stdio.h>
#include "sio.h"

/** Defines **/

#define	SC_GENERAL_SHORT_DATA_SIZE	256
#define	SC_GENERAL_EXTENDED_DATA_SIZE	65536
#define	SC_GENERAL_MAX_DATA_SIZE	65536

/* Reader (main types) */

#define SC_READER_UNKNOWN		0
#define SC_READER_DUMBMOUSE		1
#define SC_READER_TOWITOKO		2
#define	SC_READER_CTAPI			3
#define	SC_READER_VENDOR1		100
#define	SC_READER_VENDOR2		101
#define	SC_READER_VENDOR3		102

/* Cards */

#define SC_CARD_UNKNOWN				0x00
#define SC_CARD_MULTIFLEX_3K		0x01 /* 3kB EEPROM */
#define SC_CARD_MULTIFLEX_8K		0x02 /* 8kB EEPROM, more features */
#define SC_CARD_CRYPTOFLEX			0x08
#define SC_CARD_CRYPTOFLEX_DES		0x09 /* full DES option */
#define SC_CARD_CRYPTOFLEX_KEYGEN	0x0A /* full DES / RSA Key generation */
#define SC_CARD_CYBERFLEX			0x10
#define	SC_CARD_PAYFLEX				0x18
#define SC_CARD_GPK4000_S			0x20
#define SC_CARD_GPK4000_SP			0x21 /* "privacy" */
#define	SC_CARD_GPK2000_S			0x28
#define	SC_CARD_GPK2000_SP			0x29 /* "privacy" */
#define	SC_CARD_MPCOS_EMV_1B		0x30 /* 1 Byte data units */
#define	SC_CARD_MPCOS_EMV_4B		0x31 /* 4 Byte data units */
#define	SC_CARD_VENDOR1				0xF0
#define	SC_CARD_VENDOR2				0xF1
#define	SC_CARD_VENDOR3				0xF2

/* Direction */

#define SC_DIR_IN	0
#define SC_DIR_OUT	1

/* Card status */

#define	SC_CARD_STATUS_UNKNOWN	0x00
#define	SC_CARD_STATUS_PRESENT	0x01	/* Card present */
#define	SC_CARD_STATUS_CHANGED	0x02	/* Card changed */

/* T=1 checksum */

#define	SC_T1_CHECKSUM_LRC	0x00
#define	SC_T1_CHECKSUM_CRC	0x01

/* Protocols */

#define	SC_PROTOCOL_UNKNOWN	0xFF
#define SC_PROTOCOL_T0		0x00
#define SC_PROTOCOL_T1		0x01
#define SC_PROTOCOL_T14		0x0E
#define SC_PROTOCOL_I2C		0x10
#define SC_PROTOCOL_2WIRE	0x20
#define SC_PROTOCOL_3WIRE	0x40

/* Exit codes */

#define	SC_EXIT_OK					0
#define	SC_EXIT_UNKNOWN_ERROR		1	/* Error code for everything */
#define	SC_EXIT_IO_ERROR			2
#define	SC_EXIT_NOT_SUPPORTED		3
#define	SC_EXIT_NO_ACK				4
#define	SC_EXIT_BAD_ATR				5
#define	SC_EXIT_PROBE_ERROR			6
#define	SC_EXIT_BAD_CHECKSUM		7
#define	SC_EXIT_TIMEOUT				8
#define	SC_EXIT_CARD_CHANGED		9
#define	SC_EXIT_NO_CARD				10
#define	SC_EXIT_NOT_IMPLEMENTED		11
#define	SC_EXIT_CMD_TOO_SHORT		12
#define	SC_EXIT_MALLOC_ERROR		13
#define	SC_EXIT_BAD_SW				14
#define	SC_EXIT_BAD_PARAM			15
#define	SC_EXIT_NO_SLOT				16
#define	SC_EXIT_LIB_ERROR			17

/* APDU Classes */

/*
 * Class 1: lc=0, le=0
 * Class 2 Short: lc=0, le<=256
 * Class 3 Short: lc<=255, le=0
 * Class 4 Short: lc<=255, le<=256
 * Class 2 Extended: lc=0, le<=65536
 * Class 3 Extended: lc<=65535, le=0
 * Class 4 Extended: lc<=65535, le<=65536
 *
 * T=0: Class 4 becomes Class 3
 */

#define	SC_APDU_CLASS_NONE			0
#define	SC_APDU_CLASS_1				1
#define	SC_APDU_CLASS_2_SHORT		2
#define	SC_APDU_CLASS_3_SHORT		3
#define	SC_APDU_CLASS_4_SHORT		4
#define	SC_APDU_CLASS_2_EXT			5
#define	SC_APDU_CLASS_3_EXT			6
#define	SC_APDU_CLASS_4_EXT			7

/* Status words */

#define	SC_SW_OK					0
#define	SC_SW_DATA_AVAIL			1
#define	SC_SW_UNKNOWN				2
#define	SC_SW_BAD_CLA				3
#define	SC_SW_BAD_INS				4
#define	SC_SW_BAD_AUTH				5
#define	SC_SW_BAD_LENGTH			6
#define	SC_SW_BAD_P1P2				7
#define	SC_SW_FILE_ERROR			8
#define	SC_SW_INVALID_FILE			9
#define	SC_SW_END_OF_FILE			10
#define	SC_SW_FORMAT_ERROR			11
#define	SC_SW_MEMORY_ERROR			12
#define	SC_SW_NOT_FOUND				13
#define	SC_SW_EXPECTED_LEN			14
#define	SC_SW_INVALID_ACCESS		15
#define	SC_SW_BLOCKED				16
#define	SC_SW_NO_CHALLENGE			17
#define SC_SW_AUTH_LEFT				18	/* Bad authentication. Tries left. */
#define SC_SW_WRONG_CONTEXT			19
#define	SC_SW_INVALID_FUCTION		20
#define SC_SW_PURSE_ERROR			21
#define	SC_SW_WRONG_CONDITION		22

/* Command Capabilities */

#define	SC_COMMAND_APPENDREC	0x0001
#define	SC_COMMAND_ERASEBIN		0x0002
#define	SC_COMMAND_GETCHALL		0x0004
#define	SC_COMMAND_GETRESP		0x0008
#define	SC_COMMAND_READBIN		0x0010
#define	SC_COMMAND_READREC		0x0020
#define	SC_COMMAND_SELECTFILE	0x0040
#define	SC_COMMAND_UPDATEBIN	0x0080
#define	SC_COMMAND_UPDATEREC	0x0100
#define	SC_COMMAND_VERIFY		0x0200

/** Structs **/

/* Reader info */

typedef struct sc_reader_info {
	int			major;	/* main reader type */
	int			minor;	/* minor reader type */
	int			slot;	/* slot number (ICC1:1, ICC2:2, ...) */
	int			etu;	/* etu in us */
	SIO_INFO	*si;	/* serial port handle (Towitoko) */
	WORD		ctn;	/* cardterminal number (CT-API) */
} SC_READER_INFO;

/* T=1 info */

typedef struct sc_t1_info {
	BYTE	nad;		/* NAD */
	int		ns;			/* N(S) */
	int		nr;			/* N(R) */
	int		ifsc;		/* Information Filed Size Card */
	int		ifsd;		/* Information Filed Size Device */
	int		cwt;		/* Character Waiting Time in etu -11 etu */
	int		bwt;		/* Block Waiting Time in us */
	int		rc;			/* Redundancy Check (LRC/CRC) */
} SC_T1_INFO;

/* Card Info */

typedef struct sc_card_info {
	int		type;		/* Card type */
	int		status;		/* Card Status */
	BYTE	atr[32];	/* ATR */
	int		atrlen;		/* Length of ATR */
	int		protocol;	/* Used protocol */
	BOOLEAN	direct;		/* Direct convention */
	int		wwt;		/* Work Waiting Time in etu */
	SC_T1_INFO	t1;		/* for T=1 */
	int		getrsp[5];	/* GET RESPONSE Header */
	int		memsize;	/* EEPROM size (I2C,2W,3W) */
} SC_CARD_INFO;

/* APDU */

typedef struct sc_apdu {
	int		class;		/* APDU class */
	BYTE	*cmd;		/* C-APDU */
	int		cmdlen;		/* length of C-APDU */
	BYTE	*rsp;		/* R-APDU */
	int		rsplen;		/* length of R-APDU */
} SC_APDU;

/* Data block */

typedef struct sc_data {
	int		len;		/* data length */
	int		offset;		/* data offset, e.g. from begining of card */
	BYTE	*data;		/* data */
} SC_DATA;

/* Reverse String */
void SC_General_ReverseString( BYTE *data, int len);

/* Create reader */
SC_READER_INFO *SC_General_NewReader( int type, int slot );

/* Remove reader */
void SC_General_FreeReader( SC_READER_INFO **cr );

/* Create card */
SC_CARD_INFO *SC_General_NewCard( );

/* Remove card */
void SC_General_FreeCard( SC_CARD_INFO **ci );

/* Create APDU */
SC_APDU *SC_General_NewAPDU( );

/* Remove APDU */
void SC_General_FreeAPDU( SC_APDU **apdu );
\end{verbatim}

%----------------------------------------------------------------------

\subsection{sc\_reader.h}

\begin{verbatim}
#include "sc_general.h"

/* Initialize reader */
int SC_Reader_Init( SC_READER_INFO *cr, char *param );

/* Shutdown reader */
int SC_Reader_Shutdown( SC_READER_INFO *cr );

/* Activate card */
int SC_Reader_Activate( SC_READER_INFO *cr);

/* Deactivate card */
int SC_Reader_Deactivate( SC_READER_INFO *cr);

/* Get card status */
int SC_Reader_CardStatus( SC_READER_INFO *cr, SC_CARD_INFO *ci);

/* Reset card and read ATR */
int SC_Reader_ResetCard( SC_READER_INFO *cr, SC_CARD_INFO *ci );

/* Send and process T=0 command */
int SC_Reader_T0( SC_READER_INFO *cr, SC_CARD_INFO *ci, SC_APDU *apdu );

/* Transmit APDU with protocol T=1 */
int SC_Reader_T1( SC_READER_INFO *cr, SC_CARD_INFO *ci, SC_APDU *apdu );

/* Transmit APDU */
int SC_Reader_SendApdu( SC_READER_INFO *cr, SC_CARD_INFO *ci, SC_APDU *apdu );
\end{verbatim}

%----------------------------------------------------------------------

\subsection{sc\_smartcard.h}

\begin{verbatim}
#include "sc_general.h"

/* Determine the card type based on the ATR and fill data in ci */
int SC_Smartcard_GetCardType( SC_CARD_INFO *ci );

/* Process ATR and write results into ci */
int SC_Smartcard_ProcessATR( SC_CARD_INFO *ci );

/* Process status word */
int SC_Smartcard_ProcessSW( SC_CARD_INFO *ci, SC_APDU *apdu, int *status,
	int *number );

/* Fill card data in ci */
int SC_Smartcard_GetCarddata( SC_CARD_INFO *ci );

/* Common Commands */

int SC_Smartcard_Cmd_AppendRec( SC_READER_INFO *cr,
	SC_CARD_INFO *ci, BYTE *sw, BYTE *rec, int reclen );
int SC_Smartcard_Cmd_EraseBin( SC_READER_INFO *cr,
	SC_CARD_INFO *ci, BYTE *sw, int offset, int datalen );
int SC_Smartcard_Cmd_GetChall( SC_READER_INFO *cr,
	SC_CARD_INFO *ci, BYTE *sw, BYTE *challenge, int *len );
int SC_Smartcard_Cmd_GetResp( SC_READER_INFO *cr,
	SC_CARD_INFO *ci, BYTE *sw, BYTE *resp, int *resplen );
int SC_Smartcard_Cmd_ReadBin( SC_READER_INFO *cr,
	SC_CARD_INFO *ci, BYTE *sw, int offset, BYTE *data, int *datalen );
int SC_Smartcard_Cmd_ReadRec( SC_READER_INFO *cr,
	SC_CARD_INFO *ci, BYTE *sw, BYTE recnum, BYTE *rec, int *reclen );
int SC_Smartcard_Cmd_SelectFile( SC_READER_INFO *cr,
	SC_CARD_INFO *ci, BYTE *sw, int fid );
int SC_Smartcard_Cmd_UpdateBin( SC_READER_INFO *cr,
	SC_CARD_INFO *ci, BYTE *sw, int offset, BYTE *data, BYTE datalen );
int SC_Smartcard_Cmd_UpdateRec( SC_READER_INFO *cr,
	SC_CARD_INFO *ci, BYTE *sw, BYTE recnum, BYTE *rec, BYTE reclen );
int SC_Smartcard_Cmd_Verify( SC_READER_INFO *cr,
	SC_CARD_INFO *ci, BYTE *sw, BYTE *pin, int pinlen );
\end{verbatim}

%----------------------------------------------------------------------

\section{Device Independent Functions}

\subsection{SC\_General\_ReverseString}

\subsubsection*{Description}

This function reverses the bits in each byte of a byte array.
(Used for inverse convention.)

\subsubsection*{Parameters}

\begin{verbatim}
void SC_General_ReverseString( BYTE *data, int len)
\end{verbatim}

\begin{description}
\item[BYTE *data:] Byte array
\item[int len:] Length of byte array
\end{description}

\subsubsection*{Called Functions}

None.

%----------------------------------------------------------------------

\subsection{SC\_General\_NewReader}

\subsubsection*{Description}

Creates pointer to SC\_READER\_INFO and initializes it.

\subsubsection*{Parameters}

\begin{verbatim}
SC_READER_INFO *SC_General_NewReader( int type, int slot )
\end{verbatim}

\begin{description}
\item[int type:] Major card reader type (see sc\_general.h)
\item[int slot:] Card slot in reader. (Counted from 1.)
\end{description}

\begin{table}[h!]
\caption{Card slots}
\begin{center}
\begin{tabular}{|l|c|} \hline
Reader & Slots \\ \hline \hline
CT-API & varying \\ \hline
Dumbmouse & 1 \\ \hline
Towitoko & 1 \\ \hline
\end{tabular}
\end{center}
\end{table}

\subsubsection*{Called Functions}

None.

%----------------------------------------------------------------------

\subsection{SC\_General\_FreeReader}

\subsubsection*{Description}

Frees SC\_READER\_INFO pointer.

\subsubsection*{Parameters}

\begin{verbatim}
void SC_General_FreeReader( SC_READER_INFO **cr )
\end{verbatim}

\begin{description}
\item[SC\_READER\_INFO **cr:] Pointer to SC\_READER\_INFO pointer
\end{description}

\subsubsection*{Called Functions}

None.

%----------------------------------------------------------------------

\subsection{SC\_General\_NewCard}

\subsubsection*{Description}

Creates pointer to SC\_CARD\_INFO and initializes it.

\subsubsection*{Parameters}

\begin{verbatim}
SC_CARD_INFO *SC_General_NewCard( )
\end{verbatim}

\subsubsection*{Called Functions}

None.

%----------------------------------------------------------------------

\subsection{SC\_General\_FreeCard}

\subsubsection*{Description}

Frees SC\_CARD\_INFO pointer.

\subsubsection*{Parameters}

\begin{verbatim}
void SC_General_FreeCard( SC_CARD_INFO **ci )
\end{verbatim}

\begin{description}
\item[SC\_CARD\_INFO **ci:] Pointer to SC\_CARD\_INFO pointer
\end{description}

\subsubsection*{Called Functions}

None.

%----------------------------------------------------------------------

\subsection{SC\_General\_NewAPDU}

\subsubsection*{Description}

Creates pointer to SC\_CARD\_INFO and initializes it.

\subsubsection*{Parameters}

\begin{verbatim}
SC_APDU *SC_General_NewAPDU( )
\end{verbatim}

\subsubsection*{Called Functions}

None.

%----------------------------------------------------------------------

\subsection{SC\_General\_FreeAPDU}

\subsubsection*{Description}

Frees SC\_APDU pointer.

\subsubsection*{Parameters}

\begin{verbatim}
void SC_General_FreeAPDU( SC_APDU **apdu )
\end{verbatim}

\begin{description}
\item[SC\_APDU **apdu:] Pointer to SC\_APDU pointer
\end{description}

\subsubsection*{Called Functions}

None.

%----------------------------------------------------------------------

\subsection{SC\_Reader\_Init (Same as reader specific function)}

\subsubsection*{Description}

Initializes reader.

\subsubsection*{Parameters}

\begin{verbatim}
int SC_Reader_Init( SC_READER_INFO *cr, char *param )
\end{verbatim}

\begin{description}
\item[SC\_READER\_INFO *cr:] Pointer to SC\_READER\_INFO struct
\item[char *param:] Parameter string
\end{description}

\begin{table}[h!]
\caption{Init parameters}
\begin{center}
\begin{tabular}{|l|c|} \hline
Reader & Parameter \\ \hline \hline
CT-API & Port number \\ \hline
Dumbmouse & Serial port, e.g. /dev/ttyS1 or COM2 \\ \hline
Towitoko & Serial port, e.g. /dev/ttyS1 or COM2 \\ \hline
\end{tabular}
\end{center}
\end{table}

\subsubsection*{Called Functions}

\begin{itemize}
\item SC\_Ctapi\_Init
\item SC\_Dumbmouse\_Init
\item SC\_Towitoko\_Init
\end{itemize}

%----------------------------------------------------------------------

\subsection{SC\_Reader\_Shutdown (Same as reader specific function)}

\subsubsection*{Description}

Shuts reader down.

\subsubsection*{Parameters}

\begin{verbatim}
int SC_Reader_Shutdown( SIO_INFO *si , SC_READER_INFO *cr )
\end{verbatim}

\begin{description}
\item[SC\_READER\_INFO *cr:] Pointer to SC\_READER\_INFO struct
\end{description}

\subsubsection*{Called Functions}

\begin{itemize}
\item SC\_Ctapi\_Shutdown
\item SC\_Dumbmouse\_Shutdown
\item SC\_Towitoko\_Shutdown
\end{itemize}

%----------------------------------------------------------------------

\subsection{SC\_Reader\_Activate (Same as reader specific function)}

\subsubsection*{Description}

Activate card and supply voltage.

\subsubsection*{Parameters}

\begin{verbatim}
int SC_Reader_Activate( SC_READER_INFO *cr)
\end{verbatim}

\begin{description}
\item[SC\_READER\_INFO *cr:] Pointer to SC\_READER\_INFO struct
\end{description}

\subsubsection*{Called Functions}

\begin{itemize}
\item SC\_Ctapi\_Activate
\item SC\_Dumbmouse\_Activate
\item SC\_Towitoko\_Activate
\end{itemize}

%----------------------------------------------------------------------

\subsection{SC\_Reader\_Deactivate (Same as reader specific function)}

\subsubsection*{Description}

Deactive card and cut off voltage. Eject card if possible.

\subsubsection*{Parameters}

\begin{verbatim}
int SC_Reader_Deactivate( SC_READER_INFO *cr)
\end{verbatim}

\begin{description}
\item[SC\_READER\_INFO *cr:] Pointer to SC\_READER\_INFO struct
\end{description}

\subsubsection*{Called Functions}

\begin{itemize}
\item SC\_Ctapi\_Deactivate
\item SC\_Dumbmouse\_Deactivate
\item SC\_Towitoko\_Deactivate
\end{itemize}

%----------------------------------------------------------------------

\subsection{SC\_Reader\_CardStatus (Same as reader specific function)}

\subsubsection*{Description}

Sets ci->status according to card status. (See sc\_general.h)

\subsubsection*{Parameters}

\begin{verbatim}
int SC_Reader_CardStatus( SC_READER_INFO *cr, SC_CARD_INFO *ci )
\end{verbatim}

\begin{description}
\item[SC\_READER\_INFO *cr:] Pointer to SC\_READER\_INFO struct
\item[SC\_CARD\_INFO *ci:] Pointer to SC\_CARD\_INFO struct
\end{description}

\subsubsection*{Called Functions}

\begin{itemize}
\item SC\_Ctapi\_CardStatus
\item SC\_Dumbmouse\_CardStatus
\item SC\_Towitoko\_CardStatus
\end{itemize}

%----------------------------------------------------------------------

\subsection{SC\_Reader\_ResetCard (Same as reader specific function)}

\subsubsection*{Description}

Resets Smartcard and reads ATR (ci->atr, ci->atrlen).

\subsubsection*{Parameters}

\begin{verbatim}
int SC_Reader_ResetCard( SC_READER_INFO *cr, SC_CARD_INFO *ci )
\end{verbatim}

\begin{description}
\item[SC\_READER\_INFO *cr:] Pointer to SC\_READER\_INFO struct
\item[SC\_CARD\_INFO *ci:] Pointer to SC\_CARD\_INFO struct
\end{description}

\subsubsection*{Called Functions}

\begin{itemize}
\item SC\_Ctapi\_ResetCard
\item SC\_Dumbmouse\_ResetCard
\item SC\_Towitoko\_ResetCard
\end{itemize}

%----------------------------------------------------------------------

\subsection{SC\_Reader\_T0 (Same as reader specific function)}

\subsubsection*{Description}

Sends T=0 tpdus and writes response in apdu->rsp/apdu->rsplen.

Answer for Class 4 APDUs must me read with an extra Get Response.

\subsubsection*{Parameters}

\begin{verbatim}
int SC_Reader_T0( SC_READER_INFO *cr, SC_CARD_INFO *ci, SC_APDU *apdu )
\end{verbatim}

\begin{description}
\item[SC\_READER\_INFO *cr:] Pointer to SC\_READER\_INFO struct
\item[SC\_CARD\_INFO *ci:] Pointer to SC\_CARD\_INFO struct
\item[SC\_APDU *apdu:] Pointer to SC\_APDU struct
\end{description}

\subsubsection*{Called Functions}

\begin{itemize}
\item SC\_Ctapi\_T0
\item SC\_Dumbmouse\_T0
\item SC\_Towitoko\_T0
\end{itemize}

%----------------------------------------------------------------------

\subsection{SC\_Reader\_T1 (Same as reader specific function)}

\subsubsection*{Description}

Sends T=1 commands and writes response in apdu->rsp/apdu->rsplen.

\subsubsection*{Parameters}

\begin{verbatim}
int SC_Reader_T1( SC_READER_INFO *cr, SC_CARD_INFO *ci, SC_APDU *apdu )
\end{verbatim}

\begin{description}
\item[SC\_READER\_INFO *cr:] Pointer to SC\_READER\_INFO struct
\item[SC\_CARD\_INFO *ci:] Pointer to SC\_CARD\_INFO struct
\item[SC\_APDU *apdu:] Pointer to SC\_APDU struct
\end{description}

\subsubsection*{Called Functions}

\begin{itemize}
\item SC\_Ctapi\_T1
\item SC\_Dumbmouse\_T1
\item SC\_Towitoko\_T1
\end{itemize}

%----------------------------------------------------------------------

\subsection{SC\_Reader\_SendApdu (Same as reader specific function)}

\subsubsection*{Description}

Sends APDU to card and reads response. Correct protocol (T=0 or T=1)
is automagically selected. Get Response is done for T=0.

For doing correctly T=0 it is required to fill apdu->getrsp,
e.g. by calling SC\_Smartcard\_GetCardType.

\subsubsection*{Parameters}

\begin{verbatim}
int SC_Reader_SendApdu( SC_READER_INFO *cr, SC_CARD_INFO *ci,
 SC_APDU *apdu )
\end{verbatim}

\begin{description}
\item[SC\_READER\_INFO *cr:] Pointer to SC\_READER\_INFO struct
\item[SC\_CARD\_INFO *ci:] Pointer to SC\_CARD\_INFO struct
\item[SC\_APDU *apdu:] Pointer to SC\_APDU struct
\end{description}

\subsubsection*{Called Functions}

\begin{itemize}
\item SC\_Ctapi\_SendApdu
\item SC\_Dumbmouse\_SendApdu
\item SC\_Towitoko\_SendApdu
\end{itemize}

%----------------------------------------------------------------------

\subsection{SC\_Smartcard\_GetCardType}

\subsubsection*{Description}

Detects card according to ci->atr and calls SC\_Smartcard\_GetCarddata
to fill in other fields in ci, e.g. ci->getrsp. Prepares ci for
SC\_Reader\_SendApdu.

\subsubsection*{Parameters}

\begin{verbatim}
int SC_Smartcard_GetCardType( SC_CARD_INFO *ci )
\end{verbatim}

\begin{description}
\item[SC\_CARD\_INFO *ci:] Pointer to SC\_CARD\_INFO struct
\end{description}

\subsubsection*{Called Functions}

\begin{itemize}
\item SC\_Smartcard\_GetCarddata
\end{itemize}

%----------------------------------------------------------------------

\subsection{SC\_Smartcard\_ProcessATR}

\subsubsection*{Description}

Analyses ATR and writes data from it to ci.

\subsubsection*{Parameters}

\begin{verbatim}
int SC_Smartcard_ProcessATR( SC_CARD_INFO *ci )
\end{verbatim}

\begin{description}
\item[SC\_CARD\_INFO *ci:] Pointer to SC\_CARD\_INFO struct
\end{description}

\subsubsection*{Called Functions}

None.

%----------------------------------------------------------------------

\subsection{SC\_Smartcard\_ProcessSW (Same as reader specific function)}

\subsubsection*{Description}

Processes last two bytes in ci->rsp and writes status (see SC\_SW\_*
in sc\_general.h) to status. If status is SC\_SW\_DATA\_AVAIL number
contains the number of bytes of data to fetch with a Get Response.

\subsubsection*{Parameters}

\begin{verbatim}
int SC_Smartcard_ProcessSW( SC_CARD_INFO *ci, SC_APDU *apdu,
 int *status, int *number )
\end{verbatim}

\begin{description}
\item[SC\_CARD\_INFO *ci:] Pointer to SC\_CARD\_INFO struct
\item[SC\_APDU *apdu:] Pointer to SC\_APDU struct
\item[int *status:] Status of command (see SC\_SW\_* in sc\_general.h)
\item[int *number:] Contains number of bytes to fetch
\end{description}

\subsubsection*{Called Functions}

\begin{itemize}
\item SC\_Cryptoflex\_ProcessSW
\item SC\_Gpk4000\_ProcessSW
\item SC\_Multiflex\_ProcessSW
\end{itemize}

%----------------------------------------------------------------------

\subsection{SC\_Smartcard\_GetCapability (Same as reader specific function)}

\subsubsection*{Description}

Returns status word with bit flags which show the standard commands
(SC\_Smartcard\_Cmd\_*) the card supports.

\subsubsection*{Parameters}

\begin{verbatim}
int SC_Multiflex_GetCapability( SC_CARD_INFO *ci );
\end{verbatim}

\begin{description}
\item[SC\_CARD\_INFO *ci:] Pointer to SC\_CARD\_INFO struct
\end{description}

\subsubsection*{Called Functions}

\begin{itemize}
\item SC\_Cryptoflex\_GetCapability
\item SC\_Gpk4000\_GetCapability
\item SC\_Multiflex\_GetCapability
\end{itemize}

%----------------------------------------------------------------------

\subsection{SC\_Smartcard\_GetCarddata (Same as reader specific function)}

\subsubsection*{Description}

Sets data in ci according to ci->type.

\subsubsection*{Parameters}

\begin{verbatim}
int SC_Smartcard_GetCarddata( SC_CARD_INFO *ci )
\end{verbatim}

\begin{description}
\item[SC\_CARD\_INFO *ci:] Pointer to SC\_CARD\_INFO struct
\end{description}

\subsubsection*{Called Functions}

\begin{itemize}
\item SC\_Cryptoflex\_GetCarddata
\item SC\_Gpk4000\_GetCarddata
\item SC\_Multiflex\_GetCarddata
\end{itemize}

%----------------------------------------------------------------------

\section{Device Dependent Functions (Selection)}

Best documentation for device dependent functions are probably the
technical manuals (Manuals from the manufacturer, multiflex.tex in
this directory or SCDK). Here are a few examples of smart card specific
commands to show the general structure.

\subsection{SC\_Cryptoflex\_Cmd\_CreateFile}

\subsubsection*{Description}

Creates a directory or elementary file in the selected directory
file.

\subsubsection*{Parameters}

\begin{verbatim}
int SC_Cryptoflex_Cmd_CreateFile( SC_READER_INFO *cr, SC_CARD_INFO *ci,
 BYTE *sw, BYTE *fid, int flen, BYTE ftype, BYTE init, BYTE status,
 BYTE reclen, BYTE recnum, BYTE *acond, BYTE *akeys )
\end{verbatim}

\begin{description}
\item[SC\_READER\_INFO *cr:] Pointer to SC\_READER\_INFO struct
\item[SC\_CARD\_INFO *ci:] Pointer to SC\_CARD\_INFO struct
\item[BYTE *sw:] Pointer to status word (BYTE sw[2])
\item[BYTE *fid:] Pointer to file identifier (BYTE fid[2])
\item[int flen:] Length of file to create
\item[BYTE init:] 0x00: do not initialize file, 0xFF: do initialize file
\item[BYTE status:] 0x00: blocked, 0x01: unblocked
\item[BYTE reclen:] Length of records for cyclic and fixed-length record files
\item[BYTE recnum:] Number of records for cyclic and fixed-length record files
\item[BYTE *acond:] Pointer to array of access conditions (Directory, Create,
   Delete, Read, Update, Seek, Increase, Decrease) (BYTE acond[4])
\item[BYTE *akeys:] Pointer to array of access keys (BYTE akeys[3])
\end{description}

\subsubsection*{Called Functions}

\begin{itemize}
\item SC\_Reader\_T0
\end{itemize}

%----------------------------------------------------------------------
%
%\subsection{}
%
%\subsubsection*{Description}
%
%\subsubsection*{Parameters}
%
%\begin{verbatim}
%\end{verbatim}
%
%\begin{description}
%\item[SC\_READER\_INFO *cr:] Pointer to SC\_READER\_INFO struct
%\item[SC\_CARD\_INFO *ci:] Pointer to SC\_CARD\_INFO struct
%\item[SC\_APDU *apdu:] Pointer to SC\_APDU struct
%\item[:]
%\item[:]
%\item[:]
%\end{description}
%
%\subsubsection*{Called Functions}
%
%\begin{itemize}
%\item
%\item
%\item
%\end{itemize}

\section{Example}

This examples misses mostly error detecting.

\subsection{Example 1: Usage of device independent functions}

\begin{verbatim}
SC_READER_INFO *cr;
SC_CARD_INFO *ci;
SC_APDU *apdu;

BYTE selfile[]={ 0xC0, 0xA4, 0x00, 0x00, 0x02, 0x3F, 0x00, 0x00 };

cr = SC_General_NewReader( SC_READER_TOWITOKO, 1 );
ci = SC_General_NewCard( );
apdu = SC_General_NewAPDU( );

SC_Reader_Init( cr, '/dev/ttyS1' );

/* Test if card present */
SC_Reader_CardStatus( cr, ci );
if( !(ci->status & SC_CARD_STATUS_PRESENT ) exit(1);

SC_Reader_Activate( cr );
SC_Reader_ResetCard( cr, ci );

/* Prepare ci for SendApdu */
SC_Smartcard_GetCardType( ci );

/* Setup APDU */
apdu->class=SC_APDU_CLASS_4_SHORT;
apdu->cmd=cmd
apdu->cmdlen=7;
apdu->rsp=malloc(SC_GENERAL_SHORT_DATA_SIZE+2);

SC_Reader_SendApdu( cr, ci, apdu );

SC_Reader_Deactivate( cr );
SC_Reader_Shutdown( cr );

SC_General_FreeReader( &cr );
SC_General_FreeCard( &ci );
SC_General_FreeAPDU( &apdu );
\end{verbatim}

\subsection{Example 1: Usage of device dependent functions}

\begin{verbatim}
SC_READER_INFO *cr;
SC_CARD_INFO *ci;

BYTE sw[2];
BYTE fid[2];
BYTE acond[4];
BYTE akeys[3];
BYTE authkey[]={0x47,0x46,0x58,0x49,0x32,0x56,0x78,0x40};
BYTE mac[8];

BYTE buffer[SC_GENERAL_SHORT_DATA_SIZE+2];
int resplen;
int i;

...

/* Select MF */
fid[0]=0x3F; fid[1]=0x00;
SC_Multiflex_Cmd_SelectFile( cr, ci, sw, fid );

/* Get MF Data */
SC_Multiflex_Cmd_GetResp( cr, ci, sw, buffer, &resplen );

printf("Response:");
for( i=0; i<resplen; i++ ) printf(" %.2X",buffer[i]);
printf("\n");

/* Get Challenge */
SC_Multiflex_Cmd_GetChall( cr, ci, sw, buffer );

/* Encrypt Challenge */
SC_Multiflex_GenerateAuth( authkey, buffer, mac );

/* External Authenticate */
SC_Multiflex_Cmd_ExtAuth( cr, ci, sw, 0x01, mac );

/* Create transparent file 0000 */
fid[0]=0x00; fid[1]=0x00;
acond[0]=0x3F; acond[1]=0x33; acond[2]=0xFF; acond[3]=0x33;
akeys[0]=0x11; akeys[1]=0xFF; akeys[2]=0x11;
SC_Multiflex_Cmd_CreateFile( cr, ci, sw, fid, 0x17,
 SC_MULTIFLEX_FILE_TRANSPARENT, 0, 1, 0, 0, acond, akeys );

...
\end{verbatim}

\end{document}


